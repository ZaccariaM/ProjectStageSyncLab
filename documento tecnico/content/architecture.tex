\section{Scelte architetturali}
\subsection{Struttura del progetto}
    Il codice del progetto è stato organizzato in due separate per avere una netta separazione 
    dei file tra il backend e il frontend. Questo scelta ha cosentito una maggiore chiarezza, 
    modularità e ottimisticamente una facilitazione nella manutenzione del codice.
    Sia il backend che il frontend sono stati sviluppati con metodi analoghi quindi verrano 
    spiegati di consenguenza prestanzo attenzione ove ci siano differenze.
    Si è deciso di adottare una archietettura a microservizi in quanto il progetto essendo 
    di dimensioni relativamente ridotte avrebbe avuto più senso utilizzare una struttura monolitica 
    per facilitarne lo sviluppo ma essendo l'architteura a microservizi adottata più di frequente 
    si è deciso di utilizzare quest'ultima per avere una migliore resa dal punto di vista didattico. 
    Infatti se si presta attenzione alla struttura sia del backend che del frontend si può notare 
    come ogni componente sia nella sua specifica cartella o che sia comunque presente una forte 
    separazione tra i vari file in base alla loro funzionalità. \\
    Per quanto riguarda il backend non si è adottato il pattern architetturale MVC in quanto 
    avrebbe inutilmente aumentato la complessità del codice, infatti si è cercato di utilizzare il più 
    possibile l'archietettura a microservizi, per i motivi già ribaditi in precedenza, che di conseguenza 
    aumenta la modularità e scalabilità del codice.\\
    Il frontend tende anche esso ad adottare l'architettura a microservizi ove possibile, tuttavia,
    nel eventualità dove la struttura a microsevizi complicava il codice inutilmente si è optati per 
    l'archietettura più adatto a tale scopo.
