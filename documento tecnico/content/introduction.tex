\section{Introduzione}
\subsection{Scopo del documento}
    Il seguente documento ha la scopo di fornire una panoramica generale del progetto,
    descrivendo le scelte architetturali e applicative effettuate durante lo sviluppo 
    del prodotto richiesto durante il tirocinio curriculare, presso Sync Lab.
\subsection{Informazioni sul progetto}
    Il progetto consiste nello sviluppo del backend e del frontend per un 
    applicazione web che consenta di interagire con un motore LLM 
    (Large Language Model) su cloud una volta effettuato l'accesso o l'eventuale 
    registratione. Il progetto si compone di due parti principali, un frontend 
    che generi un interfaccia utente per l'interazione con l'utente e un backend 
    che si occupi di gestire la comunicazione con il motore LLM e la persistenza 
    dei dati degli utenti.

\subsection{Funzionalità del progetto}
    Il backend realizzato consente di eseguire operazioni relative alla gestione 
    degli account e dei messagi inviati dagli utenti. Infatti è possibile eseguire 
    tutte le opeazioni CRUD sugli account per i messaggi invece è possibile 
    aggiungerli nel database relativo e visualizzare tutti inviati da uno 
    specifico utente. Per quanto riguarda l'interazione con il motore LLM, 
    il backend consente di comunicarci e di memorizzare le risposte generate 
    dal motore stesso. Come ulteriore implementazione tutte le richieste utilizzano 
    un sistema di autenticazione basato su JWT.\\
    Il frontend, che comunica con il backend, implmenta le funzionalità relative 
    alla registrazione e all'accesso, alla visualizzazione dei messaggi inviati 
    e alla comunicazione con il motore LLM, ovviamente gestendo il token JWT ove presente.\\
    Infine il database realizzato in PostgreSQL consente di memorizzare i dati relaivi 
    agli utenti come email, password, username; Per quanto riguarda i messaggi 
    inviati e ricevuti viene salvato il testo del messaggio, il mittente, il timestamp 
    e l'account da cui il messaggio è stato inviato o ricevuto.